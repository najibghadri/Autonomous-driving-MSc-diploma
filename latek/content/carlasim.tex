%----------------------------------------------------------------------------
\chapter{CARLA Simulator}
\label{chap:carlasim}
%----------------------------------------------------------------------------

CARLA\cite{Dosovitskiy17} is an open-source simulator for autonomous driving
research. It is written in C++ and provides a very accessible Python API to
control a lot of the simulaton execution. CARLA has been developed from the
ground up to support development, training, and validation of autonomous driving
systems. In addition to open-source code and protocols, CARLA provides open
digital assets (urban layouts, buildings, vehicles) that were created for this
purpose and can be used freely. The simulation platform supports flexible
specification of sensor suites, environmental conditions, full control of all
static and dynamic actors, maps generation and much more. It is developed by the
Barcelonian university UAB's computer vision CVC Lab supported by Intel, Toyota,
GM and others. The repository for the project is at \url{https://github.com/carla-simulator}

CARLA's mission is to create a simulator that can simulate sufficient-enought
real-world traffic scenarios so that it is more accessible for researchers like
myself to research, develop and test computer vision algorithms for
self-driving. 

It provides Scalability via a server multi-client architecture: multiple clients
in the same or in different nodes can control different actors. CARLA exposes a
powerful API that allows users to control all aspects related to the simulation,
including traffic generation, pedestrian behaviors, weathers, sensors, and much
more. Users can configure diverse sensor suites including LIDARs, multiple
cameras, depth sensors and GPS among others. Users can easily create their own
maps following the OpenDrive standard via tools like RoadRunner. CARLA provides
integration with ROS via their ROS-bridge

I used CARLA 9.8.0 in the project that was the latest at the time (2020 March
09). Carla has a primary support for Linux so I could run it easly on Ubuntu. It
requires a decent GPU otherwise the simulation is going to be slow.

It's important to Coordinate system

\section{Is a simulation enough?}
I believe the future is in part with simulations and in part with
real-world training as-well. To develop a self-driving AI from ground up it is
certanly advisable to first develop and test the algorithms in a simulation. 

Carla provides many ways