%----------------------------------------------------------------------------
\chapter{Computer vision}
\label{chap:perceptions}
%----------------------------------------------------------------------------

After collecting data from the sensors we choose we need to implement the right
algorithms to extract information from the sensor data. In this chapter I start
with explaining basics of computer vision and then move on to advanced
convolutional neural netowrks that will help our goal.

Computer Vision, often abbreviated as CV, is defined as a field of study that
seeks to develop techniques to help computers “see” and understand the content
of digital images such as photographs and videos.

The problem of computer vision appears simple because it is trivially solved by
people, even babies. Nevertheless, it largely remains an unsolved
problem based both on the limited understanding of biological vision and because
of the complexity of vision perception in a dynamic and nearly infinitely
varying physical world.

CV algorithms can be categorized into image processing and computer vision
algorithms. The goal of image processing is to produce an image which is more
advantageous for our purposes. Image processing is often used to prepare images
for further analysis . On the other hand, the goal of computer vision is to
transform the input image to a higher abstraction level, thus to extract and
extract information from the image in a more compact manner. To further
differentiate, the term machine vision is used if these methods are applied in
embedded systems (cars, robots, production machines, smartphones, etc.).

Due to recent results in machine learning-aided computer vision applications,
these algorithms form a separate domain called learning vision. The most
dynamically advancing branch of learning vision utilizes deep learning. Other
computer vision techniques and algorithms are called tradition vision.

Traditional vision applications consist of four main steps, the first of which
is the recording of images sequences. Usually there is a second step which is
image correction, where we might reduce noise or apply effects to the images. We
will see later that in our case wil will need to occlude parts of the image
where the ego car appears in order that it won't detect itself as another car.
In the third step task-relevant features are extracted, properties which can
help solve the given problem. In the last step, called decision-making, the
applied algorithm uses the available data to determine the final output.

\section{Challenges in Computer Vision}
Object detection is considered to be the most basic application of computer
vision. Rest of the other developments in computer vision are achieved by making
small enhancements on top of this. In real life, every time we, humans open our
eyes, we unconsciously detect objects.

Since it is intuitive for us, we fail to appreciate the key challenges involved
when we try to design systems similar to our eye. Some challenges for computers are:

\begin{itemize}
    \item Variations in viewpoint
    \item Difference in illumination
    \item Hidden parts of images, occulsion
    \item Background Clutter
\end{itemize}

\section{Traditional approaches}

Various techniques, other than deep learning are available enhancing computer
vision. Though, they work well for simpler problems, but as the data become huge
and the task becomes complex, they are no substitute for deep CNNs. Let’s
briefly discuss two simple approaches.

\subsection{KNN (K-Nearest Neighbours)}

In the KNN algorithm each image is matched with all images in training data. The
top K with minimum distances are selected. The majority class of those top K is
predicted as output class of the image. Various distance metrics can be used
like L1 distance (sum of absolute distance), L2 distance (sum of squares), etc.
However KNN performs poorly - qute expectedly - they have a high error rate on
complex images, because all they do is compare pixel values among other images,
without any use of image patterns.

\subsection{Linear Classifiers}

They use a parametric approach where each pixel value is considered as a
parameter. It’s like a weighted sum of the pixel values with the dimension of
the weights matrix depending on the number of outcomes. Intuitively, we can
understand this in terms of a template. The weighted sum of pixels forms a
template image which is matched with every image. This will also face difficulty
in overcoming the challenges discussed in earlier as it is difficult to design a
single template for all the different cases.

\section{Neural networks}



\section{Deep Learning}
\section{Convolutional Neural Networks}
\section{Detection and Segmentation}
\subsection{Object Classification Detection, Localization }
    (AlexNet, LeNet, VGG)
    R-CNN, Fast, Faster
    Segmentation Networks
    Mask R-CNN - Detectron2
    Detectron
    YOLO
\subsection{Segmentation}
\section{Bounding box detection and orientation}
\section{Key point detection}
\section{Voxelization }
PointNet
VoxelNet
\section{Tracking}
others
SORT
Deep Sort
\section{Lane and road detection}
Road detection
Driveable Road
Lane detection: sliding window, curve fit
\section{3D vision}
Camera model and Calibration
3D reconstruction
Stereo vision
Depth estimation
\section{Datasets}
Datasets - KITTI, MARS, COCO, Waymo, nuScenes


Basics: depth from radar/stereo cameras: I choose only cameras










