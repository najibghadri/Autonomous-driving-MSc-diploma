%----------------------------------------------------------------------------
\chapter{Conclusion}
\label{chap:conclusion}
%----------------------------------------------------------------------------

Working on this thesis has been a unique experience because the whole filed
was new to me before getting into it. Usually thesis projects require that the
student works on the same project for 4 semesters, however I took a different
road unfortunately or not. I did my previous research work in Web APplications
and Applied blockchain technology. Then I took an optional a deep learning class
and it sparked my interest for AI even more. Taking this project was a risk and
I had to learn about basic computer vision processing methods, algorithms, 3D
vision, the camera model, convolutional neural networks and deep learning and
even a little bit of game engines because of the simulator. But in the end I
learned a lot of things and I hope I can use this knowledge soon in a nice AI
company perhaps one that works on autopilots.

The final scene understanding algorithm is not a system that can be applied by
itself in a real scenaro, however it builds on the same basic ideas for scene
understanding for cars. The work of companies like Tesla and Waymo constitues
many top researchers in the field. In Hungary this market is yet in early
stages but companies like BOSCH or a smaller company like AIMotive are already
present and working on the field with a good pace.