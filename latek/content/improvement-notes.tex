%----------------------------------------------------------------------------
\chapter{Improvement notes}
\label{chap:improvement}
%----------------------------------------------------------------------------
In \autoref{chap:assumptions} I established some simplifications to the system.
In order to create a fully capable scene understanding algorithm the following
improvements are necessary.

\section{Tracking and correlation}
In order to measure the accuracy of detections (false positives, false
negatives), it is important to correlate the positive detections with the most
likely ground truth actor. This could be done by finding the closest truth actor
to each detection. There is no point in implementing more robust solutions,
because if the error is so high that there are conflicting possibilities for the
nearest possible truth object then the depth estimation is fundamentally flawed.

\section{Faster instance segmentation with Yolact++}
A new research has emerged relating instance segmentation,
YOLACT~\cite{yolact-iccv2019} and YOLACT++~\cite{yolact-plus-arxiv2019}\footnote{
Yolact++ repository \url{https://github.com/dbolya/yolact}}, that achieves
30+fps on Titan X for instance segmentation and detection. It is based on YOLO
and uses the same resnet50 model that Detectron2 uses. If this convnet achieves
the same accuracy with a higher fps than it is replaceable with Detectron2.

\section{Optimal sensor suite}
We have seen that companies use many sensors combined not only rgb cameras. In
an optimal setting I would use only one stereo camera setting to the front and
rely on radar and ultrasonic sensors for depth data.
Monodepth~\cite{monodepth17} is also an option to estimate or correct depth
however it might not be a stable method.

\section{Keypoint based detection and orientation}
Keypoint or landmark based orientation estimation would be a robust method to
determine the orientation of vehicle in an image. This is imporant in order to
determine which side is visible to the depth map, and assign the depth data to
that side of the object and reconstruct knowing this information. In the next
chapter I describe to methods I experimented with to estimate orientation.

\section{Data correction}
The percieved information must be corrected with the car's gyroscopic data,
because cameras get tilted. This is important when the road we are on or the
road ahed of us has a high difference in inclination.

\section{Lane, path and road detection}
Lane detection can be done with the prevalent methods such a Hough transform
combined with sliding window curve fit. Another possibility is to take into
account the vehicles in front and behind us and interpret their path as the
right path and regress the lane to their path. However this might lead to
uninteded results.

\section{Foreign object detection}
With the usage of sonar and radar sensors and even more so with LiDAR it is
possible to detect object on the road. However it is more robust if the
algorithm can detect when there is an object on the road independent of what it
is exactly. A solution to this would be to use road segmentation which has to
exclude segmenting the foreign object on the road creating a hole in the
segmentation. 

\section{Traffic light understanding}
Traffic light understanding is a straightforward problem to have. Detection
algorithm trained on the COCO~\cite{DBLP:journals/corr/LinMBHPRDZ14} dataset are
already able to detect traffic lights. A difficult problem to solve is if there
are multiple traffic lights visible to the camera, but even then, usually the
closest one facing towards the vehicle is the one to follow. After determining
the traffic light we have to read the current value, which is a simple image
processing procedure. Optionally if this is not enough the algorithm can be made
more robust by teaching a convnet to be able to determine the position of the
three light circles. 

\subsection{Traffic officer detection}
Detecting traffic officers could also be a useful part of the algorithm. There
are new human pose estimation algorithms that could even help in understanding
the gestures of an officer controlling the road.  

\section{Unsupervised learning methods}
One of the most exciting improvement after all improvements above have been
achieved is to research and implement Energy based models for self-driving cars,
I recommend reading the paper "A tutorial on energy-based
learning"~\cite{Lecun98gradient-basedlearning} by Yann LeCun et al.