%----------------------------------------------------------------------------
\chapter{Improvement notes}
\label{chap:improvement}
%----------------------------------------------------------------------------
In \autoref{chap:assumptions} I established some simplifications to the system.
In order to create a fully capable scene understanding algorithm the following
improvements are needed.
\section{Faster instance segmentation with Yolact++}
A new research has emerged relating instance segmentation,
YOLACT~\cite{yolact-iccv2019} and YOLACT++~\cite{yolact-plus-arxiv2019}\footnote{
Yolact++ repository \url{https://github.com/dbolya/yolact}}, that achieves
30+fps on Titan X for instance segmentation and detection. It is based on YOLO
and uses the same resnet50 model that Detectron2 uses. If this convnet achieves
the same accuracy with a higher fps than it is replaceable with Detectron2.

\section{Optimal sensor suite}
We have seen that companies use many sensors combined not only rgb cameras. In
an optimal setting I would use only one stereo camera setting to the front and
rely on radar and ultrasonic sensors for depth data. Monodepth is also an option
to estimate or correct depth however research is still ongoing and it might not
be a stable method.
\section{Data correction}
The percieved information must be corrected with the car's gyroscopic data,
because cameras get tilted Car position, tilt, velocity detection and correction,
odometric correction
\section{Tracking and correlation}
\section{Depth correction}
Size based depth correction
Parallax motion based depth correction
\subsection{Size based}
\subsection{Monodepth}
\subsection{Parallax motion}
\section{Lane, path and road detection}
Road segmentation, path  based on other actors
Drivable area reconstruction from other actors - more robust

\section{Keypoint based detection and orientation}
Orientation, keypoint detection, wheel, etc detection
\section{3D reconstruction}
Voxel reconstruction of actors
\section{Traffic light understanding}
\section{Foreign object detection}
White list based - difficult problem! (https://link.springer.com/article/10.1186/s13640-018-0261-2)

\section{Unsupervised learning methods}
One of the most exciting improvement after all improvements above have been
achieved is to research and implement Energy based models for self-driving cars,
I recommend reading the paper "A tutorial on energy-based
learning"~\cite{Lecun98gradient-basedlearning} by Yann LeCun et al.