%----------------------------------------------------------------------------
\chapter{Assumptions made and limitations}
\label{chap:assumptions}
%----------------------------------------------------------------------------

In order to simply the task of scene understanding we need
to define boundaries to measure the success of the detector. 

\section{Daylight situation}

First of all we are going to specialize to day-light situations only. This
detection with RGB cameras at night is difficult, in order to achieve that we
need other sensors such as Radar, Sonar or LiDAR. As we are only using RGB
cameras we arge going to assume that all driving situations occur in daylight.

\section{Flat plane assumption}
Another important assumption is that the driving field and landscape area is
flat. It isn't difficult to detect object that are a bit higher on the picture
but it is difficult to recognize the curvature of the plane on the image. In
case the detector can interpret curvature and the ego car is on an angled road
the angle data from the gyroscope sensors has to be take into account and
subtracted from the percieved angles. It is generally true that inorder to
recognize true information about the world the relative position and orientation
has to be taken into account.

In order to reduce this complexity,  we are going to only take into account the
objects' position on planar coordinates.

\section{Path, lane and road detection}

As described before there are many ways of detecting lane and the easiest is to
use the Hough transform and detect the lanes directly in front of the car.
However this is not a robust solution: this only gives good results in good
illumination and weather situations. It is true that most situations are like
this but there are still many unpainted roads, dirt roads or simply due to
lightning and weather the lane edges won't be clear.

One robust solution would be to take into account the vehicles in front and
behind us and interpret their path as the right path and regress the lane to
their path. 

Another solution is to take into account previously driven paths. This is the
approach Tesla takes however it is not clear how exactly.

\section{Orientation}

It would be important to 

\section{Tracking}
- No tracking
- No consistency through time
\section{Only detection}

No control - just scene understanding:
simple steer control explain